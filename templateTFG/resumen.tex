%	RESUMEN
\chapter{RESUMEN}

Estas instrucciones servir\'an de guía para la preparaci\'on de los trabajos que se presentar\'an como requisito en el proceso de graduaci\'on de la Materia Integradora de la Unidad Acad\'emica. El resumen deber\'a contener entre 150 a 250 palabras, incluyendo los siguientes cuatro componentes, descritos de forma concisa, \textbf{ \underline{comprensible y redactado en estilo impersonal}}: 1) se empieza con una breve introducci\'on, objetivos, hip\'otesis y justificaci\'on del proyecto descrito en tiempo presente; 2) un p\'arrafo del desarrollo del proyecto, donde se describir\'an brevemente los materiales, equipos, t\'ecnicas, normas etc. utilizadas en el proyecto\cite{6469518}. Esta secci\'on se redacta en tiempo pasado; 3) otro p\'arrafo de resultados donde se describen de forma concisa los resultados escritos en tiempo pasado; 4) finalmente, se presentan las conclusiones generales del proyecto en tiempo presente. Adem\'as, deber\'a incluir al menos 4 palabras clave al final del documento. Todo el resumen se presentar\'a en un s\'olo cuerpo. Utilice el contador de palabras del procesador de texto para asegurarse del tamaño del documento. 

\textbf{Palabras Clave:} Formato, Proyecto Integrador, etc. (M\'inimo 4 y m\'aximo 5 palabras)



%	ABSTRACT
\chapter{ABSTRACT}

\textit{Use english to write the same as described before in the Resumen. Use cursive fonts in this section.}

\textbf{Keywords:} 

